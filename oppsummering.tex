\documentclass{article}
\usepackage[left=3cm, right=3cm]{geometry}
\usepackage[utf8]{inputenc}
\usepackage[norsk]{babel}
\usepackage{amsmath, amssymb}
\usepackage{mathtools}
\usepackage{hyperref}
\usepackage{feynmp-auto}
\usepackage{slashed}
%\usepackage{feynmp}

\newtheorem{theorem}{Teorem}
\newtheorem{example}{Eksempel}
\renewcommand{\times}{\cdot}
\renewcommand{\L}{\mathcal{L}}
\renewcommand{\O}{\mathcal{O}}
\newcommand{\D}{\mathcal{D}}
\newcommand{\F}[1]{\mathcal{F}\{{#1}\}}
\newcommand{\KG}{\Box + m^2}
\newcommand{\T}{\mathcal{T}}
\newcommand{\I}{\mathbb{I}}
\newcommand{\bra}[1]{\langle {#1} \mathclose|}
\newcommand{\ket}[1]{\mathopen| {#1} \rangle}
\newcommand{\abs}[1]{\mathopen| {#1} \mathclose|}
\setlength{\parindent}{0em}
\setlength{\parskip}{1em}

\title{FY3464 -- Kvantefeltteori, kort fortalt}
\author{Jonas Bueie}
\date{Oppdatert: \today}

\begin{document}
\maketitle

\tableofcontents

\section{Introduksjon}
Dette "kompendiet" gir en kort og middels oppsummering av faget FY3464 -- Kvantefeltteori I, basert på pensum fra våren 2021, læreboka til Schwartz og forelesningsnotatene til Minahan.
Leses på eget ansvar.

%\section{Definisjoner}

\section{Sentrale teoremer}
\begin{theorem}[Noethers teorem]
    Når lagrangefunksjonen har en kontinuerlig symmetri, eksisterer det en strøm forbundet med symmetrien, som er bevart når bevegelseslikningene er tilfredsstilt.
\end{theorem}

Altså er
\begin{equation*}
    \frac{\partial \L}{\partial \alpha} = 0
\end{equation*}
dersom $\L$ er symmetrisk under variasjon av $\alpha$.

\begin{example}
    Når $\phi \rightarrow e^{-i \alpha}\phi$, er denne strømmen
    \begin{equation*}
        J_\mu = \Sigma_n \frac{\partial \L}{\partial_\mu (\partial \phi_n)} \frac{\delta \phi_n}{\delta \alpha}.
    \end{equation*}
\end{example}

\begin{example}
    Hvis $\L$ er translasjonsinvariant, altså at $\phi$ er symmetrisk under $\phi(x_\mu) \rightarrow \phi(x_\mu + \xi_\mu)$, finnes en noetherstrøm per dimensjon $\mu$.
    Disse strømmene er energi- ($\mu = 0$) og impulsstrømmer ($\mu = i = 1, 2, 3$), og er gitt ved energi-impulstensoren
    \begin{equation*}
        T_{\mu \nu} = \Sigma_n \frac{\partial \L}{\partial (\partial_\mu \phi)} \partial_\nu \phi_n - \eta_{\mu \nu} \L.
    \end{equation*}
\end{example}

\begin{theorem}[Wicks teorem]
    \begin{equation*}
    T{\phi_0(x_1) \cdots \phi_0(x_n)} = :\phi_0(x_1) \cdots \phi0(x_n): 
        + \, \text{alle mulige sammentrekninger}
    \end{equation*}
\end{theorem}
Disse sammentrekningene er summen av alle mulige produkter av feynmanpropagatorer $D_F(x_i, x_j)$ mellom to felter $\phi(x_i), \phi(x_j)$.
\begin{equation*}
    D_F(x_1, x_2) \cdots D_F(x_{n-1}, x_n) + \ldots,
\end{equation*}
for enhver mulig kombinasjon av to punkter fra ${x_1, \ldots, x_n}$. 
Hvert felt $\phi$ kan kun trekkes sammen med ett annet, og alle felter må inngå i én propagator. 


\section{Kvantisering}

\subsection{Førstekvantisering}

Den klassiske hamiltonfunksjonen for en harmonisk oscillator er
\begin{equation*}
    \label{harm_osc}
    H = \frac{p^2}{2m} + \frac{1}{2} m \omega^2 x^2.
\end{equation*}
Ved å \emph{kvantisere} denne likningen, oppnår vi hamiltonfunksjonen for en kvantemekanisk harmonisk oscillator.
Dette gjøres ved å la $x$ og $p$ bli operatorer $\hat{x}$ og $\hat{p}$ som tilfredsstiller
\begin{equation*}
    [\hat{x}, \hat{p}] = i.
\end{equation*}
Vi definerer deretter skapelses- og annihileringsoperatorene $a$ og $a^\dagger$ slik at
\begin{equation*}
    a^\dagger \ket{n} = \sqrt{n+1}\ket{n+1}, \quad
    a \ket{n} = \sqrt{n-1}\ket{n-1}, \quad
\end{equation*}
\begin{equation*}
    [a, a^\dagger] = 1, og 
\end{equation*}

\begin{equation}
    H = \omega\left(a^\dagger a + \frac{1}{2} \right).
    \label{H-kvant-1}
\end{equation}
Dette kalles førstekvantisering.

\subsection{Annenkvantisering}
I en feltteori består et system av uendelig mange moder $\phi$ som alle tilfredsstiller de samme bevegelseslikningene som i det klassiske tilfellet.
Hver enkelt mode tilfredsstiller da den kvantiserte hamiltonfunksjonen \eqref{H-kvant-1}, men ved å kvantisere denne enda en gang, kan vi beskrive et felt.
\begin{equation*}
    H = \int \frac{d^3p}{(2\pi)^3} \omega_p \left(a_p^\dagger a_p + \frac{1}{2} V \right),
    \label{H-kvant-2}
\end{equation*}
der $V$ er volumet til systemet og $\omega_p = \abs{\vec{p}}$.
Hamiltonfunksjonen for systemet er nå et integral over de uendelig mange modenes individuelle hamiltonfunksjoner.
Denne \emph{annenkvantiseringen} promoterer hilbertrommet til et fockrom, og endrer skapelses- og annihilasjonsoperatorenes kommutatorrelasjon til
\begin{equation*}
    [\hat{a_p}, \hat{a^\dagger_k}] = (2\pi)^3 \delta^3(\vec{p} - \vec{k}).
\end{equation*}

$a^\dagger_p$ skaper en tilstand med impuls $\vec{p}$, og $a_p$ annihilerer den.
\begin{equation*}
    a^\dagger_p \ket{0} = \frac{1}{\sqrt{2\omega_p}}\ket{p}. 
    %a \ket{p} = \frac{1}{\sqrt{2\omega_p}}\ket{0}, \quad
\end{equation*}

\section{Frie skalarfelter}
Klein-Gordonlikningen er en kvantisert versjon av den velkjente likningen for relativistisk energi
\begin{equation*}
    E = p^2c^2 + m^2c^4 \implies (\KG) \phi = 0,
\end{equation*}
der $\phi = \phi(x) = \phi(t, \vec{x})$ er et skalart (spinnløst) felt, som kan utrykkes
\begin{equation*}
    \phi(x) = 
        \int \frac{d^3p}{(2\pi)^3} \frac{1}{\sqrt{2\omega_p}} (a_p e^{-ipx} + a^\dagger e^{ipx}),
\end{equation*}
der $p = p^\mu = (\omega_p, \vec{p})$, og skapelses- og aniihilasjonsoperatorene er de samme som for den harmoniske oscillatoren.
$\phi$ er altså en samling av uavhaengige harmoniske oscillatorer, og vi tar dette som en definisjon av frie skalarfelter, med Klein-Gordonlikningen som bevegelseslikning.

Med denne definisjonen av feltene, har vi at
\begin{equation*}
    [\phi(\vec{x}), \phi(\vec{y})] = 0,
\end{equation*}
\emph{ved samme tid $x^0 = y^0$}, og 
\begin{equation*}
    [\phi(\vec{x}), \pi(\vec{y})] 
        = [\phi(\vec{x}), \partial_t \phi(\vec{y})] 
        = i \delta(\vec{x} - \vec{y}).
\end{equation*}
Denne siste kommutatoren er Heisenbergs usikkheretsrelasjon for en feiltteori:
Et felt og dets endring er ikke målbart på samme tid og sted.

\subsection{Lagrangefunksjoner}
I kvantefeltteori er lagranefunksjoner naturlige å bruke, ettersom de er lorentzinvariante.
Vi deler normalt lagrangefunksjonen i en kinetisk del og en interaksjonsdel, analogt til den kinetiske og den potensielle energien i klassisk mekanikk.
\begin{equation*}
    \L = \L_{kin} + \L_{int} 
\end{equation*}
Den kinetiske delen består \emph{alltid} av bilineære ledd -- det vil si ledd som har nøyaktig to felter, mens interaksjonsdelen består av ledd med \emph{mer enn tre} felter.

Et fritt skalarfelt har lagrangefunksjonen
\begin{equation}
    \L = \L_{kin} = \frac{1}{2} \phi^* \Box \phi + \frac{1}{2} m^2 \phi^2
    \label{eq:L_free}
\end{equation}
uten noe interaksjonsledd.

\subsection{Korrelatorer}
Feynmanpropagatoren 
\begin{equation*}
    \begin{split}
        G_F(x - y)
        &\coloneqq \langle \phi(\vec{x}), \phi(\vec{y}) \rangle
        = \bra{0} \phi(\vec{x}) \phi(\vec{y}) \ket{0}\\
        &= \int \frac{d^4k}{(2\pi)^4} \frac{1}{k^2 - m^2 + i\epsilon} e^{ik(x - y)}\\
    \end{split}
\end{equation*}
er en greenfunksjon for Klein-Gordonlikningen
\begin{equation*}
    (\KG) G_F(x-y) = i \delta^4(x - y),
\end{equation*}
og beskriver korrelasjonen mellom feltet $\phi$ i to ulike punkter $x$ og $y$, i en \emph{fri teori}.
Dette svarer til en partikkel som beveger seg med 4-impuls $k$ og masse $m$ gjennom romtiden. Nærmere bestemt:
Sannsynligheten for at partikkelen beveger seg fra $\vec{y}$ til $\vec{x}$ i løpet av tida $y^0 - x^0$.

Denne propagatoren svarer til lagrangefunksjonen for frie skalarfelt fra likninng \eqref{eq:L_free}.
Andre propagatorer er nødvendig for andre $\L_{kin}$, for eksempel har fotoner eller diracspinorer egne propagatorer.

Svært nyttig er fourirtransformasjonen til en propagator. 
I fourierrom har feynmanpropagatoren den enkle formen
\begin{equation*}
    \tilde{G}_F(x - y) = \frac{i}{k^2 - m^2 + i\epsilon}.
\end{equation*}
Videre har $G_F$ en pol i $k^2 = m^2$. 
Disse polene kansellerer når partikkelens bevegelseslikninger er oppfylt (altså når partikkelen er ,,on-shell'').
Dette er viktig for beregning av S-matriseelementer, siden polene gir bidrag til integralet i LSZ-reduksjonsformelen (se seksjon \ref{spredning}).

\section{Veiintegralet}
Veiintegralet lar oss beregne tidsordnede produkter med utgangspunkt i et systems \emph{klassiske} egenskaper.
Integralet er 
\begin{equation*}
        \bra{\Omega} T\{\phi(x_1) \cdots \phi(x_n)\}\ket{\Omega}
        = \frac{\int \D \phi \phi(x_1) \cdots \phi(x_n) e^{iS[\phi]}}{\int \D \phi e^{iS[\phi]}},
\end{equation*}
der $S$ er den klassiske virkningen til systemet,
\begin{equation*}
    S = \int dt \L[x(t), \dot{x}(t)],
\end{equation*}
og $\D\phi$ indikerer at vi integrerer over \emph{alle} mulige feltkonfigurasjoner $\phi$ som har de rette grensebetingelsene.

For å beregne veiintegralet kan vi derfinere et genererende funksjonal
\begin{equation*}
    Z[J] = \int \D \phi exp \left\{iS[\phi] + i\int d^4x J(x)\phi(x)\right\},
\end{equation*}
slik at det tidsordnede produktet kan skrives
\begin{equation*}
        \bra{\Omega} T\{\phi(x_1) \cdots \phi(x_n)\}\ket{\Omega}
        = (-i)^n \frac{1}{Z[0]} 
        \left. \frac{\partial^n Z}{\partial J(x_1) \cdots \partial J(x_n)} \right\rvert_{J=0},
\end{equation*}
som en analog til partisjonsfunksjonen i statistisk fysikk.

Med veiintegralet er kvantemekanikken feid langt under teppet, men Schwinger-Dyson-likningene -- som gir avviket mellom klassisk og kvantemekanisk feltteori -- kan utledes fra veiintegralet.
Slik kan man se at et integral over den klassiske virkningen gir kvantemekaniske resultater.

\section{Skalar perturbasjonsteori}

\subsection{Feynmandiagrammer}
\label{feynmandiagrammer}
Feynmandiagrammer er vertkøy for å utføre perturbasjonsteori.
For enhver partikkels bevegelse gjennom tid, knytter man en del av et diagram, og disse delene settes sammen til et helt diagram.
Et feynmandiagram korresponderer med et integral, som gir sannsynligheten for at interaksjonen fra diagrammet finner sted.
Man kan dermed summere opp diagrammer med like start- og sluttilstander, og finne amplitudebidraget fra de ulike interaksjonene (diagrammene).

Hvert hjørne i diagrammet bidrar med en koplingskonstant (f.eks. $\lambda$) i integralet (se rglene nedenfor), som i perturbasjonsteori typisk er liten.
Når vi rekkeutvikler et system i denne koplingskonstanten, summerer vi derfor diagrammet med økende antall hjørner. 
Jo flere hjørner, jo høyere orden er leddet, slik at mer komplekse interaksjoner (flere hjørner i diagrammet) gir mindre amplitudebidrag enn enkle interaksjoner (færre hjærner i diagrammet).

\subsection{Feynmanregler}
Feymandiagrammer og deres integraler konstrueres gjennom feynmanregler, som er ulike for ulike lagrangefunksjoner.
Feynmanreglene kan utledes enten fra veiintegralet, eller fra lagrange- eller hamiltonfunksjonene.
I impulsrom, for lagrangefunksjoner på formen , er feynmanreglene som følger:\footnote{
    Her burde vært et avsnitt om utledning av feynmanreglene.
    }
Nedenfor følger en oppsummering av feynmanreglene, med eksempler fra skalar $\lambda \phi^4$-teori, som har lagrangefunksjonen
\begin{equation}
    \label{skalar_phi4}
    \L = \L_{kin} + \L_{int} 
        = \frac{1}{2} \phi^* \Box \phi \frac{1}{2} m^2 \phi^2
        + \frac{\lambda}{4!} \phi^4.
\end{equation}
Generelt må feynmanregler utledes uavhengig for enhver teori, så jeg gir ingen garanti for at disse reglene er generelle.

\begin{itemize}
    \item Interne linjer svarer til en fri propagator, og følger fra $\L_{kin}$. 
    \begin{itemize}
        \item For $\L$ som i likning \eqref{skalar_phi4}, representeres de med feynmanpropagatoren, $\frac{i}{p^2 - m^2 + i\epsilon}$
    \end{itemize}
    \item Hjørner svarer til interaksjoner, og følger fra $\L_{int}$. 
    \begin{itemize}
        \item For $\L$ som i likning \eqref{skalar_phi4}, representeres de med faktoren $i \lambda$.
        \item Merk: Hvis interaksjonsleddene inneholder en derivert, slik som i kvanteelektrodynamikk, må koplingskonstanten ganges med impulsen til det deriverte feltet. 
        Dette kan sees ved å skrive ut interaksjonsleddene med definisjonen av feltene som inngår i teorien.
        Leseren oppfordres til årvåkenhet i møte med mer komplekse interaksjonsledd enn fra likning \eqref{skalar_phi4}.
    \end{itemize}
    \item Impuls er bevart i hvert hjørne. Impulsbevaring representeres med en faktor $\delta(\sum_i k_i - \sum_f k_f)$ for alle hjørner.    
    \item Symmetrier reoresenteres med symmetrifaktoren $\frac{1}{s}$.
    \item For å regne ut diagrammet multipliseres faktorene fra alle punktene over, og det integreres over alle ubestemte impulser.
    \item For å regne ut et sett med diagrammer, summeres alle enkeltdiagrammene.
\end{itemize}

\subsection{Regularisering}
Det meste i kvantefeltteori divergerer\footnote{Det virker i alle fall sånn.}.
For eksempel er 1-løkkediagrammet $\infty$.
Dette kan forklares med at kvantefeltteori ikke kan gi noen nyttig verdi for størrelser som ikke er fysisk observerbare, og løkker er ikke i utgangspunktet fysisk observerbare.
Vi kan derimot finne rimelige verdier for integralene, gjennom regularisering og renormalisering.
Dette innebærer (forenklet) å innføre parametre som kan fungere som målbare referanseverdier (på samme måte som vi er vant med for potensialfunksjoner), og beregne alle andre verdier med utgangspunkt i disse.

Det finnes flere metoder for å regularisere:\footnote{
    Merk at kun de to første har vært dekket i forelesninger, men også de andre kan være ganske hendige.
}
\begin{itemize}
    \item \textbf{Avkapping (,,hard cutoff``):} Definer en maksverdi for integraldomenet basert på en naturlig fysisk størrelse, og kutt integralet her.
    \item \textbf{Dimensjonell regularisering}: La $d^4k \rightarrow d^dk$, der $d = 4-\epsilon$. 
        Dette kan i mange tilfeller fjerne divergensen i et integral over propagatorer, slik at det kan løses i grensa $\epsilon \rightarrow 0$.
        Se for øvrig identitetene i Schwartz, vedlegg B for hjelp med å løse integralene.
    \item \textbf{Derivajsonsmetoden}: En rask og ikke-ideell måte å trekke ut koeffisientene i resultatet av et UV-divergerende integral (et integral som divergerer for store $k$).
    \item \textbf{Pauli-Villarsregularisering}: Innfør en fiktiv masse $\Gamma \gg m$ og legg til denne partikkelens propagator $\frac{1}{(k^2 - \Lambda^2 +i\epsilon)^2}$ i integranden.
\end{itemize}

Etter å ha regularisert et integral, kan vi renormalisere reusltatet og oppnå fysisk meningsfulle svar.

\subsection{Renormalisering}
Renormalisering er en prosess der vi definerer en referanseverdi, en \emph{renormaliseringsbetingelse} for en variabel, slik at variabelen kan uttrykkes som en funksjon av referanseverdien.
Både i skalar $\phi^3$- og $\phi^4$-teori, samt i kvanteelektrodynamikk, er det tre størrelser som renormaliseres:
Feltet/feltene ($\phi$ eller $\psi$ og $A$), massen ($m$) og ladningen ($e$, $g$ eller $\lambda$) 


\subsection{Ladningsrenormalisering}
Koplingskonstanten(e) i $\L$ (for eksempel $g, \, \lambda$ eller $e$) må renormaliseres for at et diagram skal gi meninsgfulle svar. 
I denne prosessen uttrykker vi koplingskonstanten som ei rekkeutvikling av feynmandiagrammer, og definerer en referanseverdi for en variabel (typisk impuls).

Dette gjøres ved å skrive propagatoren som ei rekkeutvikling av feynmandiagrammer.
Disse har økende orden av koplingskonstanten, som nevnt i seksjon \ref{feynmandiagrammer}.
Vi kan så uttrykke en ny, renormalisert koplingskonstant som ei rekkeutvikling av den opprinnelige.
Denne defineres ved en referanseverdi for impulsen som vi kan definere selv\footnote{Dette er sikkert ikke særlig generelt.}.
Til slutt kan propagatoren uttrykkes som ei rekkeutvikling av den nye, renormaliserte kopblingskonstanten.

\subsection{Masserenormalisering}
\begin{example}[Masserenormalisering]
    \label{ex:renormalisering}
    1-løkkediagrammet i skalar $\lambda \phi^4$-teori er
    \unitlength = 1mm
    \begin{equation}
    \begin{gathered}
        \begin{fmffile}{truncated_loop}
            \begin{fmfgraph}(30,20)
            \fmfleft{i1}
            \fmfright{o1}
            \fmf{fermion}{i1,v,o1}
            \fmf{fermion}{v,v}
            \end{fmfgraph}
        \end{fmffile}
    \end{gathered} = \Sigma_2(k) = - \frac{i\lambda}{16\pi^2} \int_0^\infty dp \frac{p^3}{p^2 + m^2} = \infty.
    \label{one_loop}
    \end{equation}
    Dette er det andre leddet i selvenergien, derav navnet $\Sigma_2$.
    Merk at leddet er trunkert -- endepunktene er fjernet, slik som når feynmandiagram konstrueres.
    Hvis vi legger på endepunktene, har vi en propagator mellom to punkter.
    Denne skal vi finne snart.

    Integralet i likning \eqref{one_loop} kan regulariseres med avkapping 
    \begin{equation*} 
    - \frac{i\lambda}{16\pi^2} \int_0^\infty dp \frac{p^3}{p^2 + m^2} = 
     = - \frac{i\lambda}{16\pi^2} \int_0^\Lambda dp \frac{p^3}{p^2 + m^2}
     \approx - \frac{i\lambda}{32\pi^2} \left(\Lambda^2 - m^2 \ln{\frac{\Lambda^2}{m^2}}\right),
    \end{equation*} 
    for en stor $\Lambda \gg m$.
    Dette er en førsteordens korreksjon til den nullte ordens propagatoren
    \begin{equation*}
    \begin{gathered}
        \begin{fmffile}{truncated_line}
            \begin{fmfgraph}(30,15)
            \fmfleft{i1}
            \fmfright{o1}
            \fmf{fermion}{i1,o1}
            \end{fmfgraph}
        \end{fmffile}
    \end{gathered} = \frac{i}{(k^2 - m_0^2 + i\epsilon)},
    \end{equation*}
    med energi $k^2 = m^2$. 
    Det vil si at 1-løkkediagrammet representerer en vei mellom de samme grensebetingelsene, men med ett hjørne -- én interaksjon.
    Korreksjonen representerer en korreksjon i energien, $\omega \rightarrow \omega + \Delta \omega$, men siden $\omega^2 = m^2$, er dette ekvivalent med å behandle korreksjonen som en massekorreksjon
    \begin{equation*}
        m_0^2 = Z_m m_R^2 = m_R^2 + \delta_m m_R^2,
    \end{equation*}
    og vi renormaliserer bølgefunksjonene (løsningene på bevegelseslikningene) til
    \begin{equation*}
        \phi^R = \frac{1}{\sqrt{Z_2}} \phi^0, \, Z_2 = 1 + \delta_2
    \end{equation*}
    $\delta_m = \infty$ og $\delta_2 = \infty$ kalles kontraledd\footnote{
        Minhan har definert kontraleddene på en litt annen måte enn Schwartz.
    },
    og er begge av orden $\O(\lambda)$.
    De er nøklene til å fjerne uendeligheten i integralet vårt.

    Vi kjenner nå energien til to propagatorer: $\O(\lambda^0)$ og $\O(\lambda^1)$.
    Summen av disse er energien \emph{opp til første orden i $\lambda$} for en partikkels bevegelse fra $x$ til $y$ i $\lambda \phi^4$-teori.
    La oss skrive ut begge to, og summere dem:

    Den nullte ordens propagatoren -- med endepunkter -- kan skrives som en rekkeutvikling i $\delta_m$
    \begin{equation*}
    \begin{split}
    \begin{gathered}
        \begin{fmffile}{no_loop}
            \begin{fmfgraph}(30,10)
            \fmfleft{i1}
            \fmfright{o1}
            \fmf{fermion}{i1,o1}
            \fmfdot{i1,o1}
            \end{fmfgraph}
        \end{fmffile}
    \end{gathered}
        &= \frac{1}{Z_2} \frac{i}{(k^2 - m_0^2 + i\epsilon)} \\
        &= \frac{1}{1 + \delta_2} \frac{i}{k^2 - m_R^2} \frac{1}{\frac{\delta_m m_R^2}{k^2 - m_R^2} + 1}  \\
        &\simeq (1+ \delta_2) \left[ \frac{i}{k^2 - m_R^2} + \frac{i}{k^2 - m_R^2} (i \delta_m m_R^2) \frac{i}{k^2 - m_R^2} + \O(\lambda^2) \right] \\
        &= \frac{i}{k^2 - m_R^2} + \frac{i}{k^2 - m_R^2} [i (\delta_2 (k^2 - m_R^2) + \delta_m m_R^2)] \frac{i}{k^2 - m_R^2} + \O(\lambda^2),
    \end{split}
    \end{equation*}
    og den første ordens propagatoren er
    \begin{equation*}
    \begin{split}
    \begin{gathered}
        \begin{fmffile}{one_loop}
            \begin{fmfgraph}(30,15)
            \fmfleft{i1}
            \fmfright{o1}
            \fmf{fermion}{i1,v,o1}
            \fmf{fermion}{v,v}
            \fmfdot{i1,v,o1}
            \end{fmfgraph}
        \end{fmffile}
    \end{gathered} 
        &= \frac{i}{k^2 - m^2} \Sigma_2(k) \frac{i}{k^2 - m^2} \\
    \end{split}
    \end{equation*}
    Vi ser at når disse legges sammen, får vi den totale, renormaliserte \emph{effektive propagatoren}.
    \begin{equation*}
        \begin{split}
        iG^R(k) &=  \langle \frac{\phi(x)}{\sqrt{Z_2}} \frac{\phi(y)}{\sqrt{Z_2}} \rangle = \frac{1}{Z_2} G^{bar}(k) \\
        &= \frac{i}{k^2 - m_R^2}
        + \frac{i}{k^2 - m_R^2} \{
            i [\delta_2 k^2 - (\delta_2 + \delta_m) m_R^2 - \Sigma_2(k)]
        \} \frac{i}{k^2 - m_R^2} 
        + \O(\lambda^2),
        \end{split}
    \end{equation*}
    Som nevnt, er hensikten med å introdusere kontraleddene $\delta_m$ og $\delta_2$ å fjerne divergensen fra løkken.
    Dette kan vi nå gjøre, ved å finne kontraleddene slik at
    \begin{equation*}
            \delta_2 k^2 - (\delta_2 + \delta_m) m_R^2 - \Sigma_2(k) = 0,
    \end{equation*}
    altså
    \begin{equation*}
            \delta_2 k^2 - (\delta_2 + \delta_m) m_R^2 = \Sigma_2(k) 
            = - \frac{i\lambda}{32\pi^2} \left(\Lambda^2 - m^2 \ln{\frac{\Lambda^2}{m^2}}\right).
    \end{equation*}
    Ettersom $\Sigma_2(k)$ er uavhengig av $k^2$, må $\delta_2 = 0$.
    Dermed har vi bestemt kontraleddene:\footnote{Det finnes flere mulige måter å velge $\delta_m$ på. Ved å trekke fra eller legge til endelige ledd i kontraleddene, kan man oppnå ulike uttrykk for massen, dog med samme fysiske betydning. Det sentrale er at den uendelige delen av selvenergien $\Sigma_2(k)$ \emph{må} fjernes.
    De ulike valgmulighetene kalles subtraksjonsskjemaer.}
    \begin{equation*}
        \delta_2 = 0, \quad
        \delta_m 
            = - \frac{i\lambda}{32\pi^2} \left(\frac{\Lambda^2}{m^2} - \ln{\frac{\Lambda^2}{m^2}}\right),
    \end{equation*}
    og følgelig den reduserte massen $m_r$.
\end{example}

Hvis vi setter inn den regulariserte massen i lagrangefunksjonen fra eksempel \ref{ex:renormalisering} får vi
\begin{equation*}
    \begin{split}
    \L &= \frac{1}{2} \phi^* \Box \phi 
        - \frac{1}{2} m^2 \phi^2 
        + \frac{\lambda}{4!} \phi^4 \\
    \L_R &= \frac{1}{2} \phi_R^* \Box \phi_R 
        - \frac{1}{2} m_R^2 \phi_R^2 
        - \frac{1}{2} \delta_m m_R^2 \phi^2
        + \frac{\lambda}{4!} \phi_R^4. \\
    \end{split}
\end{equation*}
Renormaliseringen innebærer altså at vi legger til uendelige kontraledd i $\L$, slik at vi, når vi senere bergner integraler fra den nye lagrangefunksjonen $\L_R$, får endelige løsninger.
Regulariseringsparametre, slik som $\Lambda$, er ufysiske parametre som faller bort fra fysisk observerbare størrelser.

\subsubsection{Men hvilken masse er \emph{massen}?}
$m_0 = \sqrt{m_R^2 + \delta_m m_R^2}$ i eksempel \ref{ex:renormalisering} er nå per definisjon uendelig, men $m_R$ er ikke det.
Er denne den faktiske massen til partikkelen?
Nei, ikke egetnlig.
Den fysiske massen er definert som \emph{polmassen} til partikkelens proagator, som vi nå har renormalisert. 
Dette er altså den massen $m_P^2$ som er slik at 
\begin{equation*}
    \begin{split}
        iG^R(k) 
        &= \frac{i}{k^2 - m_R^2 - \delta_m m_R^2} \\
        &= \frac{i}{k^2 - m_R^2 + \Sigma_R(k)}
    \end{split}
\end{equation*}
divergerer.
Det vil si når
\begin{equation*}
    k^2 = m_P^2 = m_R^2 - \Sigma_R(m_P).
\end{equation*}
Slik finner vi den fysiske massen til partikkelen.

\section{Diracfelter}
Diraclikningen
\begin{equation*}
    (i \gamma^\mu \partial_\mu - m) \psi = 0
\end{equation*}
er bevegelseslikningen som følger fra diraclagrangefunksjonen
\begin{equation*}
    \L = \bar{\psi} (i \gamma^\mu \partial_\mu - m) \psi.
\end{equation*}
%Likningen kan tenkes på som en relativistisk generalisering av schrödingerlikningen, og et sentralt poeng her, er at denne generaliseringen 
som beskriver spinorer $\psi = \begin{pmatrix} \psi_L \\ \psi_R \\ \end{pmatrix}$. 
Et sentralt resultat er at disse må ha halvtallig spinn for at lorentzinvarians skal være bevart.

\subsection{Løsningen på diraclikningen}
\label{løsning_på_diraclikningen}
Diraclikningen immpliserer at Klein-Gordon-likningen stemmer:
\begin{equation*}
    (\partial + m) (i \gamma^\mu \partial_\mu - m) \psi 
    = (\partial^2 - m^2) \psi
    = 0.
\end{equation*}
Dermed må spinorer løse Klein-Gordonlikningnen, og spinorene må kunne skrives som planbølger
\begin{equation}
    \label{diracfelt}
    \psi_s(x) = \int \frac{d^3 p}{(2\pi)^3} u_s(p) e^{-ipx},
\end{equation}
med $p_0 > 0$, ettersom Klein-Gordon er en bølgelikning.
Vi definerer også antipartikkelløsningen
\begin{equation}
    \label{diracfelt_anti}
    \bar{\psi_s}(x) = \int \frac{d^3 p}{(2\pi)^3} v_s(p) e^{+ipx}.
\end{equation}

$u_s$ og $v_s$ er spinorer som gir polariseringsretningen til henholdsvis partikler og antipartikler.
Vi ønsker å uttrykke disse ved hjelp av spinorene $\xi_s$ og $\eta_s$, for 
$s \in \{1,2\} = \{\uparrow, \downarrow\}$, der vi velger 
$\xi_\uparrow = \begin{pmatrix*} 1 \\ 0 \end{pmatrix*}$ og 
$\xi_\downarrow = \begin{pmatrix*} 0 \\ 1 \end{pmatrix*}$, og tilsvarende for $\eta$.

Vi gjør dette ved å løse diraclikningen i weylbasis.
Denne likningen, som finnes ved å multiplisere ut diraclikningen for $\gamma$-matrisene i weylbasis, er
\begin{equation*}
    \begin{pmatrix*}
        -m & i\sigma^\mu \partial_mu \\
        i\sigma^\mu \partial_mu & -m \\
    \end{pmatrix*}
    \begin{pmatrix*}
        \psi_L \\
        \psi_R \\
    \end{pmatrix*}
    = 0.
\end{equation*}
I et referansesystem der partikkelen står stille, slik at $p^\mu = (m,0,0,0)$, blir spinorene da
\begin{equation*}
    \begin{pmatrix*}
        -1 & 1 \\
        1 & -1 \\
    \end{pmatrix*}
    u_s(p)
    = 0
\quad \text{og} \quad
    \begin{pmatrix*}
        -1 & -1 \\
        -1 & -1 \\
    \end{pmatrix*}
    v_s(p)
    = 0.
\end{equation*}
Hvis vi nå lar $p^\mu \rightarrow (E,0,0,p_z)$, med $E^2 = p_0^2 = \vec{p^2} + m^2$, får vi at 
\begin{equation*}
    \vec{p} \cdot \sigma = 
    \begin{pmatrix*}
        E - p_z & 0 \\
        0 & E + p_z \\
    \end{pmatrix*}
\quad \text{og} \quad
    \vec{p} \cdot \bar{\sigma} = 
    \begin{pmatrix*}
        E - p_z & 0 \\
        0 & E + p_z \\
    \end{pmatrix*},
\end{equation*}
med $\bar{\sigma} \coloneqq (\I, -\vec{\sigma})$ slik at den nye løsningen for spinorene blir 
\begin{equation*}
    u_s(p) =
    \begin{bmatrix*}
        \begin{pmatrix*}
            \sqrt{E - p_z} & 0 \\
            0 & \sqrt{E + p_z} \\
        \end{pmatrix*} \xi_s \\
        \begin{pmatrix*}
            \sqrt{E + p_z} & 0 \\
            0 & \sqrt{E - p_z} \\
        \end{pmatrix*} \xi_s
    \end{bmatrix*}
\quad \text{og} \quad
    v_s(p) =
    \begin{bmatrix*}
        \begin{pmatrix*}
            \sqrt{E - p_z} & 0 \\
            0 & \sqrt{E + p_z} \\
        \end{pmatrix*} \eta_s \\
        \begin{pmatrix*}
            -\sqrt{E + p_z} & 0 \\
            0 & -\sqrt{E - p_z} \\
        \end{pmatrix*} \eta_s
    \end{bmatrix*},
\end{equation*}
eller på en mer kompakt form
\begin{equation*}
    u_s(p) =
        \begin{pmatrix*}
            \sqrt{p\cdot\sigma} \xi_s \\
            \sqrt{p\cdot\bar{\sigma}} \xi_s
        \end{pmatrix*}
\quad \text{og} \quad
    v_s(p) =
        \begin{pmatrix*}
            \sqrt{p\cdot\sigma} \eta_s \\
            -\sqrt{p\cdot\bar{\sigma}} \eta_s
        \end{pmatrix*}.
\end{equation*}

\subsection{Kvantisering av diracfeltet og diracpropagatoren}
Som en videreføring av det frie feltet, uttrykkes ved å førstekvantisere diracfeltet fra seksjon \ref{løsning_på_diraclikningen}, likning \eqref{diracfelt} og \eqref{diracfelt_anti}
\begin{equation*}
    \begin{split}
    \psi(x) &= 
        \int \frac{d^3p}{(2\pi)^3} \frac{1}{\sqrt{2\omega_p}} (a^s_p u_s(p) e^{-ipx} + b^{s\dagger}_p v_s(p) e^{ipx}), \\
    \bar{\psi}(x) &= 
        \int \frac{d^3p}{(2\pi)^3} \frac{1}{\sqrt{2\omega_p}} (a^{s\dagger}_p u_s(p) e^{-ipx} + b^s_p v_s(p) e^{ipx}).
    \end{split}
\end{equation*}

Diracpropagatoren er
\begin{equation*}
    \tilde{S}_F(x - y) = \langle\psi_\alpha(x) \psi_\beta(y) \rangle = \frac{i}{\slashed{k} - m + i\epsilon}.
\end{equation*}

\section{Spredning}
\label{spredning}
For å beskrive spredning mellom partikler bruker vi S-matrisen 
\begin{equation*}
    S_{fi} = \bra{f}S\ket{i},
\end{equation*}
der $\ket{i}$ og $\bra{f}$ er tilstandene før og etter vekselvirkningen, altså ved $t= -\infty$ og $t= +\infty$.
S-matrisen er knyttet til spredningstverrsnittet $d\sigma$ (som er kjent fra klassisk mekanikk) gjennom relasjonen 
\begin{equation*}
    d\sigma 
        = \frac{| \bra{f} S \ket{i} | ^2}{\langle f | f \rangle \langle i | i \rangle} d\Pi,
\end{equation*}
der $d\Pi = \prod_j \frac{V}{(2\pi)^3} d^3 p_j$ er et impulselement for sluttilstanden $\ket{f}$.
Hvis vi definterer \emph{overgangsmatrisen} $\T$ slik at $S = \I + i\T$ -- og stoler blindt på Minahan -- kan vi uttrykke sannsynligheten $P_{f\leftarrow i}$ for en overgang fra $\ket{i}$ til $\ket{f}$ som
\begin{equation*}
    P_{fi} = P_{f\leftarrow i}
        = \frac{| \bra{f} \T \ket{i} | ^2}{\langle f | f \rangle \langle i | i \rangle}.
\end{equation*}
$\T = S - \I$ er den delen av $S$ som svarer til interaksjon, mens diagonalen i $S$ svarer til en fri teori, som ikke spres.

For å beregne S-matriseelementene, er LSZ-reduksjonsformelen nyttig.

\begin{theorem}[LSZ-reduksjonsformelen]
    \label{LSZ}
    Spreningsmatrisen (S-matrisen) kan uttrykkes
    \begin{equation*}
        \langle f \mathopen| S \mathclose| i \rangle 
            = \prod_{j=0}^{n} \left[ i \int d^4x_j e^{\pm i p_j x_j}(\Box_j + m^2) \right]
            \times \langle \Omega \mathopen| 
                T{\phi(x_i) \phi(x_2) \cdots \phi(x_n)} 
            \mathclose| \Omega \rangle,
    \end{equation*}
    der $-i$ og $+i$ i integranden svarer til henholsdvis start- og sluttilstander for systemet.
\end{theorem}

LSZ-formelen er et produkt av foruiertransformasjoner, $\int d^4x e^{-px}(\KG)$, av Klein-Gordon-operatoren, 
For asymptotiske tilstander ($t = \pm \infty$), er $(\Box + m^2)\phi = 0$, så kun selve interaksjonen (som skjer ved endelig $t$) bidrar til S-matrisen.
Videre er feynmanpropagatorene proporsjonale med $\frac{1}{-p^2 + m^2}$, slik at S-matrisen har poler i $p^2 - m^2 = \F{\KG}$
Dette betyr at kun leddene i tidsordningsoperatoren som har $p_i^2 = m^2$ for \emph{alle} $p_i$ det vil si kun de leddene som består av partikler på skallet.
LSZ-reduksjonsformelen reduserer med andre ord tidsordningsoperatoren til kun de tilstandene som oppfyller bevegelseslikningene, nemlig de asymptotiske tilstandene som vi ønsker å beregne.

Her er det verdt å legge merke til at feynmandiagrammer beskriver netopp vekselvirkning mellom partikler, som er opphavet til spredning.
Veiintegralet kan benyttes til å beregne LSZ-reduksjonsformelen, slik at esensielt alle verktøy og teorier som er definert i faget er nødvendige for å kunne beregne nettopp hvordan partikler påvirker hverandre:
Hvordan spredning foregår.

\end{document}

